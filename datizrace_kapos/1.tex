% Options for packages loaded elsewhere
\PassOptionsToPackage{unicode}{hyperref}
\PassOptionsToPackage{hyphens}{url}
%
\documentclass[
]{article}
\usepackage{amsmath,amssymb}
\usepackage{iftex}
\ifPDFTeX
  \usepackage[T1]{fontenc}
  \usepackage[utf8]{inputenc}
  \usepackage{textcomp} % provide euro and other symbols
\else % if luatex or xetex
  \usepackage{unicode-math} % this also loads fontspec
  \defaultfontfeatures{Scale=MatchLowercase}
  \defaultfontfeatures[\rmfamily]{Ligatures=TeX,Scale=1}
\fi
\usepackage{lmodern}
\ifPDFTeX\else
  % xetex/luatex font selection
\fi
% Use upquote if available, for straight quotes in verbatim environments
\IfFileExists{upquote.sty}{\usepackage{upquote}}{}
\IfFileExists{microtype.sty}{% use microtype if available
  \usepackage[]{microtype}
  \UseMicrotypeSet[protrusion]{basicmath} % disable protrusion for tt fonts
}{}
\makeatletter
\@ifundefined{KOMAClassName}{% if non-KOMA class
  \IfFileExists{parskip.sty}{%
    \usepackage{parskip}
  }{% else
    \setlength{\parindent}{0pt}
    \setlength{\parskip}{6pt plus 2pt minus 1pt}}
}{% if KOMA class
  \KOMAoptions{parskip=half}}
\makeatother
\usepackage{xcolor}
\usepackage[margin=1in]{geometry}
\usepackage{color}
\usepackage{fancyvrb}
\newcommand{\VerbBar}{|}
\newcommand{\VERB}{\Verb[commandchars=\\\{\}]}
\DefineVerbatimEnvironment{Highlighting}{Verbatim}{commandchars=\\\{\}}
% Add ',fontsize=\small' for more characters per line
\usepackage{framed}
\definecolor{shadecolor}{RGB}{248,248,248}
\newenvironment{Shaded}{\begin{snugshade}}{\end{snugshade}}
\newcommand{\AlertTok}[1]{\textcolor[rgb]{0.94,0.16,0.16}{#1}}
\newcommand{\AnnotationTok}[1]{\textcolor[rgb]{0.56,0.35,0.01}{\textbf{\textit{#1}}}}
\newcommand{\AttributeTok}[1]{\textcolor[rgb]{0.13,0.29,0.53}{#1}}
\newcommand{\BaseNTok}[1]{\textcolor[rgb]{0.00,0.00,0.81}{#1}}
\newcommand{\BuiltInTok}[1]{#1}
\newcommand{\CharTok}[1]{\textcolor[rgb]{0.31,0.60,0.02}{#1}}
\newcommand{\CommentTok}[1]{\textcolor[rgb]{0.56,0.35,0.01}{\textit{#1}}}
\newcommand{\CommentVarTok}[1]{\textcolor[rgb]{0.56,0.35,0.01}{\textbf{\textit{#1}}}}
\newcommand{\ConstantTok}[1]{\textcolor[rgb]{0.56,0.35,0.01}{#1}}
\newcommand{\ControlFlowTok}[1]{\textcolor[rgb]{0.13,0.29,0.53}{\textbf{#1}}}
\newcommand{\DataTypeTok}[1]{\textcolor[rgb]{0.13,0.29,0.53}{#1}}
\newcommand{\DecValTok}[1]{\textcolor[rgb]{0.00,0.00,0.81}{#1}}
\newcommand{\DocumentationTok}[1]{\textcolor[rgb]{0.56,0.35,0.01}{\textbf{\textit{#1}}}}
\newcommand{\ErrorTok}[1]{\textcolor[rgb]{0.64,0.00,0.00}{\textbf{#1}}}
\newcommand{\ExtensionTok}[1]{#1}
\newcommand{\FloatTok}[1]{\textcolor[rgb]{0.00,0.00,0.81}{#1}}
\newcommand{\FunctionTok}[1]{\textcolor[rgb]{0.13,0.29,0.53}{\textbf{#1}}}
\newcommand{\ImportTok}[1]{#1}
\newcommand{\InformationTok}[1]{\textcolor[rgb]{0.56,0.35,0.01}{\textbf{\textit{#1}}}}
\newcommand{\KeywordTok}[1]{\textcolor[rgb]{0.13,0.29,0.53}{\textbf{#1}}}
\newcommand{\NormalTok}[1]{#1}
\newcommand{\OperatorTok}[1]{\textcolor[rgb]{0.81,0.36,0.00}{\textbf{#1}}}
\newcommand{\OtherTok}[1]{\textcolor[rgb]{0.56,0.35,0.01}{#1}}
\newcommand{\PreprocessorTok}[1]{\textcolor[rgb]{0.56,0.35,0.01}{\textit{#1}}}
\newcommand{\RegionMarkerTok}[1]{#1}
\newcommand{\SpecialCharTok}[1]{\textcolor[rgb]{0.81,0.36,0.00}{\textbf{#1}}}
\newcommand{\SpecialStringTok}[1]{\textcolor[rgb]{0.31,0.60,0.02}{#1}}
\newcommand{\StringTok}[1]{\textcolor[rgb]{0.31,0.60,0.02}{#1}}
\newcommand{\VariableTok}[1]{\textcolor[rgb]{0.00,0.00,0.00}{#1}}
\newcommand{\VerbatimStringTok}[1]{\textcolor[rgb]{0.31,0.60,0.02}{#1}}
\newcommand{\WarningTok}[1]{\textcolor[rgb]{0.56,0.35,0.01}{\textbf{\textit{#1}}}}
\usepackage{graphicx}
\makeatletter
\newsavebox\pandoc@box
\newcommand*\pandocbounded[1]{% scales image to fit in text height/width
  \sbox\pandoc@box{#1}%
  \Gscale@div\@tempa{\textheight}{\dimexpr\ht\pandoc@box+\dp\pandoc@box\relax}%
  \Gscale@div\@tempb{\linewidth}{\wd\pandoc@box}%
  \ifdim\@tempb\p@<\@tempa\p@\let\@tempa\@tempb\fi% select the smaller of both
  \ifdim\@tempa\p@<\p@\scalebox{\@tempa}{\usebox\pandoc@box}%
  \else\usebox{\pandoc@box}%
  \fi%
}
% Set default figure placement to htbp
\def\fps@figure{htbp}
\makeatother
\setlength{\emergencystretch}{3em} % prevent overfull lines
\providecommand{\tightlist}{%
  \setlength{\itemsep}{0pt}\setlength{\parskip}{0pt}}
\setcounter{secnumdepth}{-\maxdimen} % remove section numbering
\usepackage{bookmark}
\IfFileExists{xurl.sty}{\usepackage{xurl}}{} % add URL line breaks if available
\urlstyle{same}
\hypersetup{
  pdftitle={Butkevica\_datizrace\_kapseta},
  hidelinks,
  pdfcreator={LaTeX via pandoc}}

\title{Butkevica\_datizrace\_kapseta}
\author{}
\date{\vspace{-2.5em}2025-09-12}

\begin{document}
\maketitle

\section{Datu avots}\label{datu-avots}

Man neizdevas atras daubāzi, kur būtu pieejami atsevišķi ieraksti ar
precīžu vecumu vai dzimšanas un mirstības gadu, savukārt manuāla
kopēšanas no \url{https://timenote.info/} neveido dabisku izlasi, tādēļ
es izmantoju datus no Pasaules \href{https://www.who.int/}{Veselības
organizācijas} (World Health Organization), kas (izvelētais fails) satur
informāciju par mirušo cilvēku sakitu no 1988 līdz 2002 gadam pa vecuma
grupam. Vecuma grupas nesakrīt pilnība ar uzdevuma apsaktītajam, tomēr,
manuprāt tas neietekme izpildes algoritmus. Dati lejupielādei pieejami
\href{https://www.who.int/data/data-collection-tools/who-mortality-database}{šeit}.

\section{Datu sakartošana}\label{datu-sakartoux161ana}

Datubaze satur informāciju par mirušo skaitu gadā, katrā vecuma grupā,
sadalot to pa dzimumiem un miršanas iemesliem. Tādēļ no sākuma sakartoju
datus. Tas nebija daļa no uzdevuma, tādēl detalizēti neskaidrošu.

\begin{Shaded}
\begin{Highlighting}[]
\NormalTok{pirmie\_dati }\OtherTok{\textless{}{-}} \FunctionTok{read.csv}\NormalTok{(}\FunctionTok{here}\NormalTok{(}\StringTok{"datizrace\_kapos"}\NormalTok{, }\StringTok{"Morticd10\_part1.csv"}\NormalTok{))}

\NormalTok{dati }\OtherTok{\textless{}{-}}\NormalTok{ pirmie\_dati[pirmie\_dati}\SpecialCharTok{$}\NormalTok{Country }\SpecialCharTok{==} \DecValTok{4186}\NormalTok{,] }\CommentTok{\# atlasu Latviju}

\NormalTok{apvienoti\_dati }\OtherTok{\textless{}{-}}\NormalTok{ dati }\SpecialCharTok{\%\textgreater{}\%}
  \FunctionTok{group\_by}\NormalTok{(Sex) }\SpecialCharTok{\%\textgreater{}\%}         \CommentTok{\# grupēju pēc dzimuma}
  \FunctionTok{summarise}\NormalTok{(}\FunctionTok{across}\NormalTok{(}\FunctionTok{starts\_with}\NormalTok{(}\StringTok{"Death"}\NormalTok{), sum, }\AttributeTok{na.rm =} \ConstantTok{TRUE}\NormalTok{))}
\end{Highlighting}
\end{Shaded}

\begin{verbatim}
## Warning: There was 1 warning in `summarise()`.
## i In argument: `across(starts_with("Death"), sum, na.rm = TRUE)`.
## i In group 1: `Sex = 1`.
## Caused by warning:
## ! The `...` argument of `across()` is deprecated as of dplyr 1.1.0.
## Supply arguments directly to `.fns` through an anonymous function instead.
## 
##   # Previously
##   across(a:b, mean, na.rm = TRUE)
## 
##   # Now
##   across(a:b, \(x) mean(x, na.rm = TRUE))
\end{verbatim}

\begin{Shaded}
\begin{Highlighting}[]
\NormalTok{apvienoti\_dati}\SpecialCharTok{$}\NormalTok{Sex }\OtherTok{\textless{}{-}} \FunctionTok{ifelse}\NormalTok{(apvienoti\_dati}\SpecialCharTok{$}\NormalTok{Sex }\SpecialCharTok{==} \DecValTok{1}\NormalTok{, }\StringTok{"Vīrietis"}\NormalTok{, }\StringTok{"Sieviete"}\NormalTok{)}


\CommentTok{\# Apvienot vecumu 0{-}4 gadi vienā intervālā, jo tie sniegti pa atsevišķiem gadiem}
\NormalTok{apvienoti\_dati}\SpecialCharTok{$}\NormalTok{Vec0\_4 }\OtherTok{\textless{}{-}} \FunctionTok{rowSums}\NormalTok{(apvienoti\_dati[, }\FunctionTok{c}\NormalTok{(}
  \StringTok{"Deaths2"}\NormalTok{, }\StringTok{"Deaths3"}\NormalTok{, }\StringTok{"Deaths4"}\NormalTok{, }\StringTok{"Deaths5"}\NormalTok{, }\StringTok{"Deaths6"}
\NormalTok{  )], }\AttributeTok{na.rm =} \ConstantTok{TRUE}\NormalTok{)}

\CommentTok{\# Izdzest liekas kolonnas}
\NormalTok{apvienoti\_dati }\OtherTok{\textless{}{-}}\NormalTok{ apvienoti\_dati }\SpecialCharTok{\%\textgreater{}\%}
  \FunctionTok{select}\NormalTok{(}\SpecialCharTok{{-}}\NormalTok{Deaths1, }\SpecialCharTok{{-}}\NormalTok{Deaths2, }\SpecialCharTok{{-}}\NormalTok{Deaths3, }\SpecialCharTok{{-}}\NormalTok{Deaths4,}
         \SpecialCharTok{{-}}\NormalTok{Deaths5, }\SpecialCharTok{{-}}\NormalTok{Deaths6,}\SpecialCharTok{{-}}\NormalTok{Deaths25, }\SpecialCharTok{{-}}\NormalTok{Deaths26)}

\CommentTok{\# Precizēt nosaukumus}
\NormalTok{apvienoti\_dati }\OtherTok{\textless{}{-}}\NormalTok{ apvienoti\_dati }\SpecialCharTok{\%\textgreater{}\%}
  \FunctionTok{rename}\NormalTok{(}
    \AttributeTok{Vec5\_9 =}\NormalTok{ Deaths7, }\AttributeTok{Vec10\_14 =}\NormalTok{ Deaths8, }\AttributeTok{Vec15\_19 =}\NormalTok{ Deaths9,}
    \AttributeTok{Vec20\_24 =}\NormalTok{ Deaths10, }\AttributeTok{Vec25\_29 =}\NormalTok{ Deaths11, }\AttributeTok{Vec30\_34 =}\NormalTok{ Deaths12,}
    \AttributeTok{Vec35\_39 =}\NormalTok{ Deaths13, }\AttributeTok{Vec40\_44 =}\NormalTok{ Deaths14, }\AttributeTok{Vec45\_49 =}\NormalTok{ Deaths15,}
    \AttributeTok{Vec50\_54 =}\NormalTok{ Deaths16, }\AttributeTok{Vec55\_59 =}\NormalTok{ Deaths17, }\AttributeTok{Vec60\_64 =}\NormalTok{ Deaths18,}
    \AttributeTok{Vec65\_69 =}\NormalTok{ Deaths19, }\AttributeTok{Vec70\_74 =}\NormalTok{ Deaths20, }\AttributeTok{Vec75\_79 =}\NormalTok{ Deaths21,}
    \AttributeTok{Vec80\_84 =}\NormalTok{ Deaths22, }\AttributeTok{Vec85\_89 =}\NormalTok{ Deaths23, }\AttributeTok{Vec90\_94 =}\NormalTok{ Deaths24}
\NormalTok{  )}

\CommentTok{\# Pareizā kolonnu secība }
\NormalTok{apvienoti\_dati }\OtherTok{\textless{}{-}}\NormalTok{ apvienoti\_dati }\SpecialCharTok{\%\textgreater{}\%}
  \FunctionTok{relocate}\NormalTok{(Vec0\_4, }\AttributeTok{.after =} \DecValTok{1}\NormalTok{)}
\end{Highlighting}
\end{Shaded}

Iegūta datu tabula satur mirušo cilvēku skaitu katrā vecumgrupā katram
dzimumam atsevišķi.

\begin{Shaded}
\begin{Highlighting}[]
\FunctionTok{head}\NormalTok{(apvienoti\_dati)}
\end{Highlighting}
\end{Shaded}

\begin{verbatim}
## # A tibble: 2 x 20
##   Sex      Vec0_4 Vec5_9 Vec10_14 Vec15_19 Vec20_24 Vec25_29 Vec30_34 Vec35_39
##   <chr>     <dbl>  <int>    <int>    <int>    <int>    <int>    <int>    <int>
## 1 Vīrietis   2422    554      504     1634     3074     3742     4850     7136
## 2 Sieviete   1938    282      292      648      704      872     1320     2184
## # i 11 more variables: Vec40_44 <int>, Vec45_49 <int>, Vec50_54 <int>,
## #   Vec55_59 <int>, Vec60_64 <int>, Vec65_69 <int>, Vec70_74 <int>,
## #   Vec75_79 <int>, Vec80_84 <int>, Vec85_89 <int>, Vec90_94 <int>
\end{verbatim}

\section{Pirmais uzdevums}\label{pirmais-uzdevums}

Aprēķināt, cik ilgi (vidēji) dzīvos cilvēks, kurš jau ir sasniedzis
noteiktu vecumu (piemēram, 65 gadus)?

Algoritma apraksts:

\begin{enumerate}
\def\labelenumi{\arabic{enumi}.}
\item
  Atlasīt tikai tās kolonnas, kur vecuma intervāls sākas no 65 gadiem
  (\texttt{Vec65\_69}, \texttt{Vec70\_74} utt.).
\item
  Aprēķināt katras vecuma grupas ``vidējo vecumu'', jo precīzu vecumu
  mēs nezinam, tādēļ šāda vējda mēgīnām līdzsvarot. Piemēram,
  \texttt{Vec65\_69} → vidējais vecums = 67 gadi.
\item
  Aprēķināt kopējo cilvēku skaitu ≥65 gadi. Sasumēt visus mirušos šajās
  vecuma grupās (pa dzimumiem).
\item
  Aprēķināt vidējo atlikušās dzīves ilgumu pēc formulas:\\
  \[
  \text{Videjais atlikusais dzives ilgums} = \frac{\sum (\text{miruso skaits} \times \text{videjais vecums grupa}) - 65 \times \text{kopejais skaits}}{\text{kopejais skaits}}
  \]
\item
  Iegustāmais rezultāts: vidējais dzīves ilgums pēc 65 gadu vecuma
  katram dzimumam.
\end{enumerate}

Sākuma definējam interesējušos intervālus un aprēķinām katrā vēcuma
intervāla vidējo vecuma vērtību.

\begin{Shaded}
\begin{Highlighting}[]
\NormalTok{vecuma\_kolonnas }\OtherTok{\textless{}{-}} \FunctionTok{c}\NormalTok{(}\StringTok{"Vec65\_69"}\NormalTok{,}\StringTok{"Vec70\_74"}\NormalTok{,}\StringTok{"Vec75\_79"}\NormalTok{,}\StringTok{"Vec80\_84"}\NormalTok{,}\StringTok{"Vec85\_89"}\NormalTok{,}\StringTok{"Vec90\_94"}\NormalTok{)}
\NormalTok{vidus\_vecums }\OtherTok{\textless{}{-}} \FunctionTok{c}\NormalTok{(}\DecValTok{67}\NormalTok{,}\DecValTok{72}\NormalTok{,}\DecValTok{77}\NormalTok{,}\DecValTok{82}\NormalTok{,}\DecValTok{87}\NormalTok{,}\DecValTok{92}\NormalTok{)  }\CommentTok{\# vidējie vecumi katrai grupai}
\end{Highlighting}
\end{Shaded}

Apreķinām vidējo atlikušas dzīves ilgumu katram dzimumam

\begin{Shaded}
\begin{Highlighting}[]
\NormalTok{rezultats1 }\OtherTok{\textless{}{-}}\NormalTok{ apvienoti\_dati }\SpecialCharTok{\%\textgreater{}\%}
  \FunctionTok{rowwise}\NormalTok{() }\SpecialCharTok{\%\textgreater{}\%}
  \FunctionTok{mutate}\NormalTok{(}
    \CommentTok{\# skaits un vecuma grupas * skaits}
    \AttributeTok{kop\_65\_plus =} \FunctionTok{sum}\NormalTok{(}\FunctionTok{c\_across}\NormalTok{(}\FunctionTok{all\_of}\NormalTok{(vecuma\_kolonnas))),}
    \AttributeTok{suma\_vidus =} \FunctionTok{sum}\NormalTok{(}\FunctionTok{c\_across}\NormalTok{(}\FunctionTok{all\_of}\NormalTok{(vecuma\_kolonnas)) }\SpecialCharTok{*}\NormalTok{ vidus\_vecums),}
    \AttributeTok{vid\_pec\_65 =}\NormalTok{ (suma\_vidus }\SpecialCharTok{{-}} \DecValTok{65} \SpecialCharTok{*}\NormalTok{ kop\_65\_plus)}\SpecialCharTok{/}\NormalTok{kop\_65\_plus}
\NormalTok{  ) }\SpecialCharTok{\%\textgreater{}\%}
  \FunctionTok{select}\NormalTok{(Sex, vid\_pec\_65)}
\end{Highlighting}
\end{Shaded}

\emph{Atbilde:}

\begin{Shaded}
\begin{Highlighting}[]
\NormalTok{rezultats1}
\end{Highlighting}
\end{Shaded}

\begin{verbatim}
## # A tibble: 2 x 2
## # Rowwise: 
##   Sex      vid_pec_65
##   <chr>         <dbl>
## 1 Vīrietis       10.4
## 2 Sieviete       14.4
\end{verbatim}

\section{Otrais uzdevums}\label{otrais-uzdevums}

Agoritms:

Ievade: tabula ar kolonnām: Sex, Vec0\_4, Vec5\_9, Vec10\_14, \ldots{}
1. Apvienot visus dzimumus, saskaitot pa vecuma intervāliem. 2.
Aprēķināt kopējo mirušo skaitu visās vecuma grupās. 3. Katram vecuma
intervālam: - atrast skaitu šajā grupā - aprēķināt proporciju = (skaits
/ kopējais) * 100 4. Izvadīt tabulu vai histogrammu ar proporcijām.

\section{Trešais uzdevums}\label{treux161ais-uzdevums}

Ievade: histogramma ar kolonnu Vecuma\_grupa un Proporcija (mirušo \% no
100) 1. Sakārtot vecuma grupas pēc kārtas (0--4, 5--9, \ldots, 90--94).
2. Inicializēt sākuma populāciju = 100. 3. Katram vecuma intervālam: a)
Paņemt proporciju (mirušo \% no 100) b) Atņemt to no ``dzīvo'' skaita c)
Saglabāt rezultātu kā ``cik dzīvo šajā vecumā'' 4. Pārrēķināt uz
procentiem (sadalījums uz 100). 5. Rezultātu var izmantot, lai uztaisītu
histogrammu ``iedzīvotāju vecuma spektrs''.

\end{document}
